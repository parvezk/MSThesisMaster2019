%
%  This is an example of how a LaTeX thesis should be formatted.  This
%  document contains chapter 1 of the thesis.
%

\chapter{CONCLUSION}
%%%%%%%% This line gets rid of the page number on the first page of text
\thispagestyle{empty}
%%%%%%%%%%%%%

%INJECT[The first paragraph in the Discussion should summarize the Results. Most readers will read the Abstract, maybe the Introduction, and then the Discussion. Write the Discussion as if it were the first thing your readers saw.]

%INJECT[Then you need 2-3 paragraphs placing your results into a broader context. What does your work mean for the major question described in the Introduction? Also, how does your work relate to other work in the field? What specifically are the similarities and differences?]

%INJECT[Finally, and this is somewhat optional, you can write another paragraph that summarizes all your findings once more]

Research in deep learning has traditionally focused on new algorithms, mathematical models, improving quality and performance or the speed of the neural network model. We have studied and investigated a lateral research direction that touches upon the social implication of the automated decision-making systems, namely, we have contributed to furthering the understanding and transparency of the decision making implemented by a trained deep neural network. 

We proposed a visual exploration tool that provides a visual explanation for the inference decision made by an image recognition system. The tool is targeted towards non-experts and helps broaden people's access to an interactive tool for deep learning. The visualization technique helps make an image recognition model more transparent by providing a visual explanation for their predictions. Our prototype helped to answer two critical questions about model inference raised in our research hypothesis: (i) why did the network think this image contained a specific object (ii) where is the object located in the picture. We used the heatmap concept that allowed better intuition about what has been learned by the network.

Further, running deep learning application entirely client-side in the browser unlocked new opportunities to add rich interaction and user experience. From a user's perspective, there is no need to install any libraries or drivers. They can directly access the application in their browser. Finally, all the user data stays on the client-side local to the user device, and this helps to maintain as a privacy-preserving application.

In summary, we have presented a novel image explanation technique that justifies the class prediction of a visual classifier. This method could be a  stepping stone and serve as a foundation for more robust model interpretability and explainable system.


