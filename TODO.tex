CONTENTS TO COVER---------

OVERALL___________________
CREDIT IMAGES USED
DEFINITION | GLOSSARY
USING LOCALIZATION AND REVANCE HEATMAP
 research question, literature review 
 identifying the weaknesses or limitations of your study and your future research plans

\textbf{INTRODUCTION}======================================

\textbf{BACKGROUND}======================================

DNN: NEURON  and ANN COMPARISON
CNN: IMAGE 1 and 2 FOOTNOTE
MODEL CONVERSION

\textbf{METHODOLOGY}=====================================
\textbf{Research Hypothesis}
- While deep neural networks learn efficient and powerful representations, they are often considered a ‘black-box’. [VISUAL EXPLANATION PAPER] What are the tools and methods 
- Visualizing the contributions of individuals nodes in complex networks

\textbf{PLEASE REFER RED HIGHLIGHTS IN PAPER CAM - approach + 2 pts}

\textbf{DESIGN GOALS} / Prototype Objective
Localization approach like CAM are highly class-discriminative, exclusively highlights the object regions, localizing target objects in scenes

As a result, important regions of the image which correspond
to any decision of interest are visualized in high-resolution
detail even if the image contains evidence for multiple possible concepts

VGG16: (CNN) that achieves top-5 accuracy of 89.5% on the ImageNet dataset that contains over 1.2 millions images across 1000 classes.

APPROACH: SENSITVITY ANALYSIS using HEATMAP Concept
Get visual interpretation of which parts of the image more most
 *    responsible for a convnet's classification decision, using the
 *    gradient-based class activation map (CAM) method.
 
 \textbf{SYSTEM DESIGN}
 \textbf{Web-based / Experimental setup}
 Remove Tensorflow.js
 - Interactive interface to visualize classes attribution graphs of a model....
 - Deployment using cross-platform, lightweight web technologies...
Infrastructure - HPC VM

 \textbf{PROTOTYPE - DeepViz User InterfaceVision Objective, Motivation}

Vision:  Visual Exploration Tool - Explanatory Graph of Attribution Map / Activations
A visual exploration tool targeted towards non-technical audience
A visual narrative of the inference process

- A visual exploration tool to explore the inference made by an image recognition model
- Provide rationale for decision made by the model
- Help users distill large, complex neural network models into compact, interactive graph visualizations

A visual exploration tool to interpret class predictions
Show builds up its understanding of images over many layers

Motivated by notions of interpretability and assessing trust in models, we evaluate DeepViz via human studies to show that they can be important tools for users to evaluate and place trust in automated systems.

Our approach is weakly supervised localization in context of CNN: where the task is to localize objects in images using only whole image/video frame class labels

Our goal is to build an interactive visualization tool for users to better
understand how neural networks build their hierarchical representation.


  The header of SUMMIT displays metadata about the image clas-
fication model being visualized, such as the model and dataset name, the number of classes, and the total number data instances within the dataset.

\textbf{RESULTS}============================================

\textbf{GITHUB CODE BASE}

For image classification, our visualizations help identify dataset bias () and lend insight into failures of current CNNs, showing that seemingly unreasonable predictions have reasonable explanations. For

We conduct user testing that show CAM explanation are class discriminative, and not only gelp  human establish trust, but also help untrained users successfully discern a stronger network from a weaker, even when both make identical predictions

TRUST: Thus our viz can help users place trust in a model that can localize better, just based on individual prediction explanations.

Viz helps accurately explain the function learned by the model.

DETECT BIAS: This experiment demonstrates that Grad-CAM can
help detect and remove biases in datasets, which is impor-tant not just for generalization, but also for fair and ethical
outcomes as more algorithmic decisions are made in society.

\textit{REFER HIGHLIGHTS FROM GRAD-CAM PAPER}

\textit{CONCLUSION PART OF THE FEATURE VIZ ARTICLE}



