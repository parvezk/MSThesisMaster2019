CONTENTS TO COVER---------

OVERALL___________________
CREDIT IMAGES USED
DEFINITION | GLOSSARY
USING LOCALIZATION AND RELEVANCE HEATMAP
 research question, literature review 
 identifying the weaknesses or limitations of your study and your future research plans

\textbf{METHODOLOGY}=====================================
\textbf{Research Hypothesis}
- While deep neural networks learn efficient and powerful representations, they are often considered a ‘black-box’. [VISUAL EXPLANATION PAPER] What are the tools and methods 
- Visualizing the contributions of individuals nodes in complex networks

\textbf{PLEASE REFER RED HIGHLIGHTS IN PAPER CAM - approach + 2 pts}

\textbf{DESIGN GOALS} / Prototype Objective
Localization approach like CAM are highly class-discriminative, exclusively highlights the object regions, localizing target objects in scenes

As a result, important regions of the image which correspond
to any decision of interest are visualized in high-resolution
detail even if the image contains evidence for multiple possible concepts

VGG16: (CNN) that achieves top-5 accuracy of 89.5% on the ImageNet dataset that contains over 1.2 millions images across 1000 classes.

APPROACH: SENSITVITY ANALYSIS using HEATMAP Concept
Get visual interpretation of which parts of the image more most
 *    responsible for a convnet's classification decision, using the
 *    gradient-based class activation map (CAM) method.


\textbf{RESULTS}============================================

\textbf{GITHUB CODE BASE}

For image classification, our visualizations help identify dataset bias () and lend insight into failures of current CNNs, showing that seemingly unreasonable predictions have reasonable explanations. For

We conduct user testing that show CAM explanation are class discriminative, and not only gelp  human establish trust, but also help untrained users successfully discern a stronger network from a weaker, even when both make identical predictions

TRUST: Thus our viz can help users place trust in a model that can localize better, just based on individual prediction explanations.

Viz helps accurately explain the function learned by the model.

DETECT BIAS: This experiment demonstrates that Grad-CAM can
help detect and remove biases in datasets, which is impor-tant not just for generalization, but also for fair and ethical
outcomes as more algorithmic decisions are made in society.

\textit{REFER HIGHLIGHTS FROM CAM PAPER}

\textit{CONCLUSION PART OF THE FEATURE VIZ ARTICLE}

EXTRACT==============


\subsection{Action Research}
Action Research is a research methodology driven by practical problems and emphasis on participatory research. It develops practically useful solutions to a real-world problem iteratively. It offers a systematic and collaborative approach to conducting research that satisfies both the need for scientific rigor and promotion of sustainable social change and has been taken up by a variety of researchers in both academic and industrial settings \cite{Hayes:2011:RAR:1993060.1993065}.

Action Research is comparative research on the effects and conditions of various forms of social action, and research effort leading to practical social action that utilizes a spiral of steps, each of which composed of a circle of planning, action, review and fact-finding about the result of the action\cite{Hayes:2011:RAR:1993060.1993065}. Action Research is not completely a method but instead its a a series of commitments and practice to observe and frame the problem through a series of principles for conducting various social inquiry.

In terms of other research approaches, Action Research differs in its ontological, epistemological, and methodological commitments \cite{edsjsr.312187519970101}. These underlying assumptions put the researcher and their association with research participants at the heart of the process of inquiry, covering all of the ways in which data are collected, analyzed, and reported and change is implemented.

Its also an interdisciplinary approach and explicitly democratic and collaborative. When conducing action research, the focus is to create research efforts with general people experiencing real problems in their everyday lives and not “for”, “about”, or “focused on” them \cite{Hayes:2011:RAR:1993060.1993065}. Thus, action research research focuses on localized solutions that are highly contextualized, with a greater emphasis on transferability than generalization. 

The action Rrsearch methodology is open-ended and iterative. The primary focus of AR is to implement action iteratively, in which action can include a policy or process change, the introduction of new technology, or other intervention, and significant measures of the work are both the quality of research results produced and the feasibility of the solution(s) that emerged. It utilizes cycles of inquiry that include planning, action, and reflection, in which the action being undertaken is continually designed and evaluated with research results emerging throughout these cycles. AR can incorporate multiple methods and welcomes the use of both qualitative and quantitative methods.

When it comes to cyclical rigor, Action Research is cyclical in nature, with an emphasis on problem formulation, design of an intervention, action (e.g., deploying the intervention), observation of the effects of the action, reflection, and then redefinition of the problem to start the cycle again. Visualizations of this process show action research as a cycle in circular format or a spiral with the circles progressing in some manner. The goal then is not to arrive at the solution to a given problem but to attempt to create a solution that is “better” in some way than previous solutions and helps the researchers and practitioners learn through the action they take

The research of interpretability in deep learning and machine learning has far reaching impact because they encourage transparency and accountability of the black box models. Action Research will be a useful method in this research as it emphasizes the knowledge produced in the context of the application. \cite{401014119781201}. It’s a distinct candidate research method when the objective is to explore theory concerning practice.

